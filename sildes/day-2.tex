\documentclass{beamer}

\mode<presentation> {
\usetheme{CambridgeUS}
\usecolortheme{seagull}
\usefonttheme{default}
}

\usepackage{graphicx}
\usepackage{booktabs}
\usepackage{ragged2e}
\usepackage{minted}
\usepackage{lipsum}
\usepackage[export]{adjustbox}

%----------------------------------------------------------------------------------------
%	My Customized Settings
%----------------------------------------------------------------------------------------

\definecolor{UniBlue}{RGB}{83,121,170}
\definecolor{DarkGray}{RGB}{90,90,90}
\definecolor{LightGray}{RGB}{150,150,150}
\definecolor{TextGreen}{RGB}{115,155,15}
\definecolor{Ocean}{RGB}{23,142,189}
\definecolor{BG}{RGB}{215,215,215}


\setbeamercolor{normal}{fg=DarkGray}
\setbeamercolor{title}{fg=UniBlue}
\setbeamercolor{frametitle}{fg=UniBlue}
\setbeamercolor{structure}{fg=UniBlue}
\setbeamercolor{normal text}{fg=DarkGray,bg=white}
\setbeamercolor{section number projected}{bg=UniBlue,fg=white}

\setbeamertemplate{itemize item}{\scriptsize\raise1.25pt\hbox{\donotcoloroutermaths$\blacktriangleright$}}
\setbeamertemplate{itemize subitem}{\tiny\raise1.5pt\hbox{\donotcoloroutermaths$\bullet$}}
\setbeamertemplate{itemize subsubitem}{\tiny\raise1.5pt\hbox{\donotcoloroutermaths$\blacksqaure$}}
\setbeamercolor{itemize item}{fg=darkred}
\setbeamercolor{itemize subitem}{fg=TextGreen}
\setbeamercolor{itemize subbody}{fg=LightGray}

\setbeamertemplate{enumerate subitem}{\insertenumlabel.\insertsubenumlabel}
\setbeamertemplate{enumerate subsubitem}{\insertenumlabel.\insertsubenumlabel.\insertsubsubenumlabel}
\setbeamertemplate{enumerate mini template}{\insertenumlabel}

\setbeamertemplate{navigation symbols}{}

\newcommand\VeryLargeFont{\fontsize{30}{15}\selectfont}
\newcommand\LargeFont{\fontsize{15}{15}\selectfont}
\newcommand\TinyFont{\fontsize{6}{6}\selectfont}

\setbeamertemplate{frametitle} {
  \nointerlineskip
  \begin{beamercolorbox}[sep=0.15cm,ht=1.3em,wd=\paperwidth]{frametitle}
    \vbox{}\vskip-2ex
    \strut\insertframetitle\strut
    \vskip-0.8ex
  \end{beamercolorbox}
}

\defbeamertemplate*{title page}{customized}[1][] {
  \centering
  \bigskip
  \bigskip
  \bigskip
  \usebeamercolor[fg]{title}\insertsubtitle\par
  \usebeamerfont{title}\inserttitle\par
  \usebeamerfont{subtitle}
  \bigskip
  \usebeamercolor[fg]{normal}
  \usebeamerfont{author}\insertauthor\par
  \usebeamerfont{institute}\insertinstitute\par
  \usebeamerfont{date}\insertdate\par
  \bigskip
  \bigskip
  \bigskip
  \bigskip
  \bigskip
  \usebeamercolor[fg]{titlegraphic}\inserttitlegraphic
}

%----------------------------------------------------------------------------------------
%	TITLE PAGE
%----------------------------------------------------------------------------------------
\title[Introduction]{An Introduction to Python}
\author{Parham Alvani}
\institute[AUT] {
  Amirkabir University of Technology \\
  \medskip
  {\small\tt parham.alvani@gmail.com}
}
\date{May 15, 2015}
\titlegraphic{\hspace*{5cm}\includegraphics[width=2cm]{figs/aut_logo.jpeg}}

\begin{document}

\begin{frame}
\titlepage
\end{frame}

%----------------------------------------------------------------------------------------
%	PRESENTATION SLIDES
%----------------------------------------------------------------------------------------

%------------------------------------------------
\section{}
\subsection{}

%------------------------------------------------
\begin{frame}[fragile]
	\frametitle{Keyboard Input}
	\begin{columns}[c]
		\begin{column}{30cm}
			\vspace{.1cm}
			\begin{itemize}
				\item The input() function reads a line from sys.stdin and \\
				 returns it with the trailing newline stripped.
			\end{itemize}
			\begin{scriptsize}
				\begin{minted}[
				bgcolor=BG,
				frame=lines,
				framesep=2mm,
				baselinestretch=1.2,
				linenos]
				{python}
				name = input("Enter your input: ")
				print("Received input is : ", name)
				\end{minted}
			\end{scriptsize}
		\end{column}
	\end{columns}
\end{frame}

%------------------------------------------------
\begin{frame}
	\frametitle{File IO}
	\begin{columns}[c]
		\begin{column}{30cm}
			\vspace{.1cm}
			\begin{itemize}
				\justifying
				\item The open Function
				\item The file Object
				\item The close() Method
				\item The write() Method
				\item The read() Method
				\item See
				\textcolor{blue}{\href{https://docs.python.org/2.4/lib/bltin-file-objects.html}{here}}
				for more details and functions.
			\end{itemize}
		\end{column}
	\end{columns}
\end{frame}

%------------------------------------------------
\begin{frame}[fragile]
	\frametitle{The open Function}
	\begin{columns}[c]
		\begin{column}{30cm}
			\vspace{.1cm}
			\begin{itemize}
				\item Before you can read or write a file, \\
				you have to open it using Python's built-in open() function.
				\item This function creates a file object, which would be utilized \\
				to call other support methods associated with it.
			\end{itemize}
			\begin{scriptsize}
				\begin{minted}[
				bgcolor=BG,
				frame=lines,
				framesep=2mm,
				baselinestretch=1.2,
				linenos]
				{python}
				file object = open(file_name [, access_mode][, buffering])
				\end{minted}
				\begin{minted}[
				bgcolor=BG,
				frame=lines,
				framesep=2mm,
				baselinestretch=1.2,
				linenos]
				{python}
				fo = open("foo.txt", "w")
				\end{minted}
			\end{scriptsize}
		\end{column}
	\end{columns}
\end{frame}


%------------------------------------------------
\begin{frame}[fragile]
	\frametitle{The write() Method}
	\begin{columns}[c]
		\begin{column}{30cm}
			\vspace{.1cm}
			\begin{itemize}
				\item The write() method writes any string to an open file.
			\end{itemize}
			\begin{scriptsize}
				\begin{minted}[
				bgcolor=BG,
				frame=lines,
				framesep=2mm,
				baselinestretch=1.2,
				linenos]
				{python}
				fileObject.write(string)
				\end{minted}
				\begin{minted}[
				bgcolor=BG,
				frame=lines,
				framesep=2mm,
				baselinestretch=1.2,
				linenos]
				{python}
				# Open a file
				fo = open("foo.txt", "wb")
				fo.write( "Python is a great language.\nYeah its great!!\n")
				
				# Close opend file
				fo.close()
				\end{minted}
			\end{scriptsize}
		\end{column}
	\end{columns}
\end{frame}

%------------------------------------------------
\begin{frame}[fragile]
	\frametitle{The read() Method}
	\begin{columns}[c]
		\begin{column}{30cm}
			\vspace{.1cm}
			\begin{itemize}
				\item The read() method reads a string from an open file.
			\end{itemize}
			\begin{scriptsize}
				\begin{minted}[
				bgcolor=BG,
				frame=lines,
				framesep=2mm,
				baselinestretch=1.2,
				linenos]
				{python}
				fileObject.read([count])
				\end{minted}
				\begin{minted}[
				bgcolor=BG,
				frame=lines,
				framesep=2mm,
				baselinestretch=1.2,
				linenos]
				{python}
				# Open a file
				fo = open("foo.txt", "r+")
				str = fo.read(10);
				print "Read String is : ", str
				# Close opend file
				fo.close()
				\end{minted}
			\end{scriptsize}
		\end{column}
	\end{columns}
\end{frame}

%------------------------------------------------
\begin{frame}[fragile]
	\frametitle{The readline() Method}
	\begin{columns}[c]
		\begin{column}{30cm}
			\vspace{.1cm}
			\begin{itemize}
				\item Read one entire line from the file.
				\item A trailing newline character is kept in the string.
				\item If the size argument is present and non-negative, \\
				it is a maximum byte count (including the trailing newline) \\
				and an incomplete line may be returned.
			\end{itemize}
			\begin{scriptsize}
				\begin{minted}[
				bgcolor=BG,
				frame=lines,
				framesep=2mm,
				baselinestretch=1.2,
				linenos]
				{python}
				fileObject.readline([size])
				\end{minted}
			\end{scriptsize}
		\end{column}
	\end{columns}
\end{frame}

%------------------------------------------------
\begin{frame}[fragile]
	\frametitle{The readlines() Method}
	\begin{columns}[c]
		\begin{column}{30cm}
			\vspace{.1cm}
			\begin{itemize}
				\item Read until EOF using readline() and return a \\
				 list containing the lines thus read.
			\end{itemize}
			\begin{scriptsize}
				\begin{minted}[
				bgcolor=BG,
				frame=lines,
				framesep=2mm,
				baselinestretch=1.2,
				linenos]
				{python}
				fileObject.readlines([sizehint])
				\end{minted}
			\end{scriptsize}
		\end{column}
	\end{columns}
\end{frame}

%------------------------------------------------
\begin{frame}[fragile]
	\frametitle{Iterators \& Generators}
	\begin{columns}[c]
		\begin{column}{30cm}
			\vspace{.1cm}
			\begin{itemize}
				\justifying
				\item There are many types of objects which can be used with a for loop.\\
				These are called \textcolor{orange}{iterable} objects.
				\item The built-in function \textcolor{orange}{iter} takes an \textcolor{orange}{iterable}
				object and returns an iterator.
			\end{itemize}
			\begin{scriptsize}
				\begin{minted}[
				bgcolor=BG,
				frame=lines,
				framesep=2mm,
				baselinestretch=1.2,
				linenos]
				{python}
				>>> x = iter([1, 2, 3])
				>>> x
				<listiterator object at 0x1004ca850>
				>>> x.next()
				1
				>>> x.next()
				2
				>>> x.next()
				3
				\end{minted}
			\end{scriptsize}
		\end{column}
	\end{columns}
\end{frame}

%------------------------------------------------
\begin{frame}[fragile]
	\frametitle{Iterators \& Generators}
	\begin{columns}[c]
		\begin{column}{30cm}
			\vspace{.1cm}
			\begin{itemize}
				\justifying
				\item Iterators are implemented as classes.
			\end{itemize}
			\begin{scriptsize}
				\begin{minted}[
				bgcolor=BG,
				frame=lines,
				framesep=2mm,
				baselinestretch=1.2,
				linenos]
				{python}
				class yrange:
				    def __init__(self, n):
				        self.i = 0
				        self.n = n
				
				    def __iter__(self):
				        return self
				
				    def next(self):
				        if self.i < self.n:
				            i = self.i
				            self.i += 1
				            return i
				        else:
				            raise StopIteration()
				\end{minted}
			\end{scriptsize}
		\end{column}
	\end{columns}
\end{frame}

%------------------------------------------------
\begin{frame}[fragile]
	\frametitle{Iterators \& Generators}
	\begin{columns}[c]
		\begin{column}{30cm}
			\vspace{.1cm}
			\begin{itemize}
				\justifying
				\item In the above case, both the iterable and iterator are the same object. Notice that the \_\_iter\_\_ method returned self
			\end{itemize}
			\begin{scriptsize}
				\begin{minted}[
				bgcolor=BG,
				frame=lines,
				framesep=2mm,
				baselinestretch=1.2,
				linenos]
				{python}
				class zrange:
				    def __init__(self, n):
				        self.n = n
				
				    def __iter__(self):
				        return zrange_iter(self.n)
				\end{minted}
			\end{scriptsize}
		\end{column}
	\end{columns}
\end{frame}

%------------------------------------------------
\begin{frame}[fragile]
	\frametitle{Iterators \& Generators}
	\begin{columns}[c]
		\begin{column}{30cm}
			\vspace{.1cm}
			\begin{scriptsize}
				\begin{minted}[
				bgcolor=BG,
				frame=lines,
				framesep=2mm,
				baselinestretch=1.2,
				linenos]
				{python}
				class zrange_iter:
				    def __init__(self, n):
				        self.i = 0
				        self.n = n
				
			 	    def __iter__(self):
				        # Iterators are iterables too.
				        # Adding this functions to make them so.
				        return self
				
				    def __next__(self):
				        if self.i < self.n:
				            i = self.i
				            self.i += 1
				            return i
				        else:
				            raise StopIteration()
				\end{minted}
			\end{scriptsize}
		\end{column}
	\end{columns}
\end{frame}

%------------------------------------------------
\begin{frame}
	\frametitle{Iterators \& Generators}
	\begin{Huge}
		\begin{center}
			\usebeamercolor[fg]{title}If both iteratable and iterator are the same object, it is consumed in a single iteration.
		\end{center}
	\end{Huge}
\end{frame}

%------------------------------------------------
\begin{frame}
	\frametitle{Modules}
	\begin{columns}[c]
		\begin{column}{30cm}
			\vspace{.1cm}
			\begin{itemize}
				\justifying
				\item A module allows you to logically organize your Python code.
				\item Grouping related code into a module makes the code easier to understand and use
			\end{itemize}
		\end{column}
	\end{columns}
\end{frame}

%------------------------------------------------
\begin{frame}[fragile]
	\frametitle{Modules}
	\begin{columns}[c]
		\begin{column}{30cm}
			\vspace{.1cm}
			\begin{scriptsize}
				\begin{minted}[
				bgcolor=BG,
				frame=lines,
				framesep=2mm,
				baselinestretch=1.2,
				linenos]
				{python}
				def print_func( par ):
				    print "Hello : ", par
				    return
				\end{minted}
				\begin{minted}[
				bgcolor=BG,
				frame=lines,
				framesep=2mm,
				baselinestretch=1.2,
				linenos]
				{python}
				# Import module support
				import support
							
				# Now you can call defined function that module as follows
				support.print_func("Zara")
				\end{minted}
			\end{scriptsize}
		\end{column}
	\end{columns}
\end{frame}

%------------------------------------------------
\begin{frame}
	\frametitle{Modules}
	\begin{columns}[c]
		\begin{column}{30cm}
			\vspace{.1cm}
			\begin{itemize}
				\justifying
				\item The current directory.
				\item Python then searches each directory in the shell variable PYTHONPATH.
				\item If all else fails, Python checks the default path. \\
				On UNIX, this default path is normally /usr/local/lib/python/.
			\end{itemize}
		\end{column}
	\end{columns}
\end{frame}

%------------------------------------------------
\begin{frame}[fragile]
	\frametitle{Socket Programming}
	\begin{columns}[c]
		\begin{column}{30cm}
			\vspace{.1cm}
			\begin{itemize}
				\justifying
				\item Python provides two levels of access to network services
				\item To create a socket, you must use the socket.socket()
				\item \textcolor{green}{socket\_family}: This is either AF\_UNIX or AF\_INET.
				\item \textcolor{green}{socket\_type}: This is either SOCK\_STREAM or SOCK\_DGRAM.
				\item \textcolor{green}{protocol}: This is usually left out, defaulting to 0.
			\end{itemize}
			\begin{scriptsize}
				\begin{minted}[
					bgcolor=BG,
					frame=lines,
					framesep=2mm,
					baselinestretch=1.2,
					linenos]
					{python}
					s = socket.socket (socket_family, socket_type, protocol=0)
				\end{minted}
			\end{scriptsize}
		\end{column}
	\end{columns}
\end{frame}

%------------------------------------------------
\begin{frame}[fragile]
	\frametitle{Socket Programming}
	\begin{columns}[c]
		\begin{column}{30cm}
			\vspace{.1cm}
			\begin{scriptsize}
				\begin{minted}[
				bgcolor=BG,
				frame=lines,
				framesep=2mm,
				baselinestretch=1.2,
				linenos]
				{python}
				import socket               # Import socket module
				
				s = socket.socket()         # Create a socket object
				host = socket.gethostname() # Get local machine name
				port = 12345                # Reserve a port for your service.
				s.bind((host, port))        # Bind to the port
				
				s.listen(5)                 # Now wait for client connection.
				while True:
				    c, addr = s.accept()     # Establish connection with client.
				    print('Got connection from %s' % addr)
				    c.send('Thank you for connecting')
				    c.close()                # Close the connection
				\end{minted}
			\end{scriptsize}
		\end{column}
	\end{columns}
\end{frame}

%------------------------------------------------
\begin{frame}
	\vspace{1cm}
	\begin{Huge}
		\begin{center}
			\usebeamercolor[fg]{title}Questions?
		\end{center}
	\end{Huge}
\end{frame}

\end{document} 