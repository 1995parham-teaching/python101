\documentclass{beamer}

\mode<presentation> {
\usetheme{CambridgeUS}
\usecolortheme{seagull}
\usefonttheme{default}
}

\usepackage{graphicx}
\usepackage{booktabs}
\usepackage{ragged2e}
\usepackage{minted}
\usepackage{lipsum}
\usepackage[export]{adjustbox}

%----------------------------------------------------------------------------------------
%	My Customized Settings
%----------------------------------------------------------------------------------------

\definecolor{UniBlue}{RGB}{83,121,170}
\definecolor{DarkGray}{RGB}{90,90,90}
\definecolor{LightGray}{RGB}{150,150,150}
\definecolor{TextGreen}{RGB}{115,155,15}
\definecolor{Ocean}{RGB}{23,142,189}
\definecolor{BG}{RGB}{215,215,215}


\setbeamercolor{normal}{fg=DarkGray}
\setbeamercolor{title}{fg=UniBlue}
\setbeamercolor{frametitle}{fg=UniBlue}
\setbeamercolor{structure}{fg=UniBlue}
\setbeamercolor{normal text}{fg=DarkGray,bg=white}
\setbeamercolor{section number projected}{bg=UniBlue,fg=white}

\setbeamertemplate{itemize item}{\scriptsize\raise1.25pt\hbox{\donotcoloroutermaths$\blacktriangleright$}}
\setbeamertemplate{itemize subitem}{\tiny\raise1.5pt\hbox{\donotcoloroutermaths$\bullet$}}
\setbeamertemplate{itemize subsubitem}{\tiny\raise1.5pt\hbox{\donotcoloroutermaths$\blacksqaure$}}
\setbeamercolor{itemize item}{fg=darkred}
\setbeamercolor{itemize subitem}{fg=TextGreen}
\setbeamercolor{itemize subbody}{fg=LightGray}

\setbeamertemplate{enumerate subitem}{\insertenumlabel.\insertsubenumlabel}
\setbeamertemplate{enumerate subsubitem}{\insertenumlabel.\insertsubenumlabel.\insertsubsubenumlabel}
\setbeamertemplate{enumerate mini template}{\insertenumlabel}

\setbeamertemplate{navigation symbols}{}

\newcommand\VeryLargeFont{\fontsize{30}{15}\selectfont}
\newcommand\LargeFont{\fontsize{15}{15}\selectfont}
\newcommand\TinyFont{\fontsize{6}{6}\selectfont}

\setbeamertemplate{frametitle} {
  \nointerlineskip
  \begin{beamercolorbox}[sep=0.15cm,ht=1.3em,wd=\paperwidth]{frametitle}
    \vbox{}\vskip-2ex
    \strut\insertframetitle\strut
    \vskip-0.8ex
  \end{beamercolorbox}
}

\defbeamertemplate*{title page}{customized}[1][] {
  \centering
  \bigskip
  \bigskip
  \bigskip
  \usebeamercolor[fg]{title}\insertsubtitle\par
  \usebeamerfont{title}\inserttitle\par
  \usebeamerfont{subtitle}
  \bigskip
  \usebeamercolor[fg]{normal}
  \usebeamerfont{author}\insertauthor\par
  \usebeamerfont{institute}\insertinstitute\par
  \usebeamerfont{date}\insertdate\par
  \bigskip
  \bigskip
  \bigskip
  \bigskip
  \bigskip
  \usebeamercolor[fg]{titlegraphic}\inserttitlegraphic
}

%----------------------------------------------------------------------------------------
%	TITLE PAGE
%----------------------------------------------------------------------------------------
\title[Introduction]{Introduction to Python}
\author{Parham Alvani}
\institute[9231058] {
  Amirkabir University of Technology \\
  \medskip
  {\small\tt parham.alvani@gmail.com}
}
\date{May 14, 2015}
\titlegraphic{\hspace*{5cm}\includegraphics[width=2cm]{figs/aut_logo.jpeg}}

\begin{document}

\begin{frame}
\titlepage
\end{frame}

%----------------------------------------------------------------------------------------
%	PRESENTATION SLIDES
%----------------------------------------------------------------------------------------

%------------------------------------------------
\section{}
\subsection{}

%------------------------------------------------
\begin{frame}
\begin{columns}
	\begin{column}{10cm}
		\vspace{2cm}
		\begin{block}{
				\centering\textcolor{darkred}{what is python...}}
				\justifying
				Python is a high-level, interpreted, interactive and object-oriented scripting language. Python is designed to be highly readable..\\
		\end{block}
	\end{column}
\end{columns}
\vspace{.75cm}
\hspace*{8.5cm}\includegraphics[width=3cm]{figs/python.jpeg}
\end{frame}

%------------------------------------------------
\begin{frame}
	\frametitle{Environment}
	\hspace*{1.5cm}\includegraphics[width=10cm]{figs/python-console-linux.png}
\end{frame}

%------------------------------------------------
\begin{frame}
	\frametitle{Variable Types}
	\begin{columns}[c]
		\begin{column}{30cm}
			\vspace{.1cm}
			\begin{itemize}
				\justifying
				\item Numbers
				\item String
				\item List
				\item Tuple
				\item Dictionary
			\end{itemize}
		\end{column}
	\end{columns}
	\vspace{.5cm}
	\hspace*{5.5cm} \includegraphics[width=5cm]{figs/python-types.jpg}
\end{frame}

%------------------------------------------------
\begin{frame}
	\frametitle{Numbers}
	\begin{columns}[c]
		\begin{column}{30cm}
			\vspace{.1cm}
			\begin{itemize}
				\justifying
				\item Number data types store numeric values.
				\item They are immutable data types, means that changing the value of \\
				 a number data type results in a newly allocated object.
			\end{itemize}
		\end{column}
	\end{columns}
\end{frame}

%------------------------------------------------
\begin{frame}
	\frametitle{Strings}
	\begin{columns}[c]
		\begin{column}{30cm}
			\vspace{.1cm}
			\begin{itemize}
				\justifying
				\item Strings are amongst the most popular types in Python.
				\item We can create them simply by enclosing characters in quotes.
				\item Python treats single quotes the same as double quotes.
			\end{itemize}
		\end{column}
	\end{columns}
\end{frame}

%------------------------------------------------
\begin{frame}
	\frametitle{Lists}
	\begin{columns}[c]
		\begin{column}{30cm}
			\vspace{.1cm}
			\begin{itemize}
				\justifying
				\item The most basic data structure in Python is the sequence.
				\item Each element of a sequence is assigned a number - its position or index.
				\item The first index is zero, the second index is one, and so forth.
			\end{itemize}
		\end{column}
	\end{columns}
	\vspace{.5cm}
	\hspace*{5.5cm} \includegraphics[width=5cm]{figs/list.jpg}	
\end{frame}

%------------------------------------------------
\begin{frame}
	\frametitle{Tuples}
	\begin{columns}[c]
		\begin{column}{30cm}
			\vspace{.05cm}
			\begin{itemize}
				\justifying
				\item A tuple is a sequence of immutable Python objects.
				\item Tuples are sequences, just like lists.
				\item The differences between tuples and lists are :
				\begin{itemize}
						\item the tuples cannot be changed unlike lists
						\item tuples use parentheses, whereas lists use square brackets.
				\end{itemize}
			\end{itemize}
		\end{column}
	\end{columns}
	\vspace{.1cm}
	\hspace*{5.5cm} \includegraphics[width=5cm]{figs/tuple.jpg}
\end{frame}

%------------------------------------------------
\begin{frame}
	\frametitle{Dictionary}
	\begin{columns}[c]
		\begin{column}{30cm}
			\vspace{.1cm}
			\begin{itemize}
				\justifying
				\item Each key is separated from its value by a colon (:)
				\item the items are separated by commas
				\item the whole thing is enclosed in curly braces.
			\end{itemize}
		\end{column}
	\end{columns}
	\vspace{.5cm}
	\hspace*{5.5cm} \includegraphics[width=5cm]{figs/dictionary.jpg}
\end{frame}

%------------------------------------------------
\begin{frame}
	\frametitle{Flow Control}
	\begin{columns}[c]
		\begin{column}{30cm}
			\vspace{.1cm}
			\begin{itemize}
				\justifying
				\item If-Then-Else
				\item For
				\item While
				\item Exceptions
			\end{itemize}
		\end{column}
	\end{columns}
	\vspace{.5cm}
	\hspace*{5.5cm} \includegraphics[width=5cm]{figs/flow-control.jpg}
\end{frame}

%------------------------------------------------
\begin{frame}
	\frametitle{Functions}
	\begin{columns}[c]
		\begin{column}{30cm}
			\vspace{.1cm}
			\begin{itemize}
				\justifying
				\item Function blocks begin with the keyword \textcolor{orange}{def}, \\
				followed by the function name and parentheses.
				\item Any input parameters or arguments should be placed \\
				within these parentheses.
				\item The first statement of a function can be an optional statement \\
				- the documentation string of the function or docstring.
			\end{itemize}
		\end{column}
	\end{columns}
	\vspace{.1cm}
	\hspace*{5cm} \includegraphics[width=5cm]{figs/functions.jpg}
\end{frame}

%------------------------------------------------
\begin{frame}[fragile]
	\frametitle{Functions}
	\begin{columns}[c]
		\begin{column}{30cm}
			\vspace{.1cm}
			\begin{scriptsize}
			  	\begin{minted}[
			  	bgcolor=BG,
			  	frame=lines,
			  	framesep=2mm,
			  	baselinestretch=1.2,
			  	linenos]
			  	{python}
		  			def square(x):
				  	    return x * x
		  		\end{minted}
		  		\begin{minted}[
		  		bgcolor=BG,
		  		frame=lines,
		  		framesep=2mm,
		  		baselinestretch=1.2,
		  		linenos]
		  		{python}
			  		def hello():
		  			    return "Hello"
			  	\end{minted}
			  	\begin{minted}[
			  	bgcolor=BG,
			  	frame=lines,
			  	framesep=2mm,
			  	baselinestretch=1.2,
			  	linenos]
			  	{python}
			  	def printme( str ):
			  	    "This prints a passed string into this function"
			  	    print str
			  	    return
			  	\end{minted}
			\end{scriptsize}
		\end{column}
	\end{columns}
\end{frame}

%------------------------------------------------
\begin{frame}
	\frametitle{Classes}
	\begin{columns}[c]
		\begin{column}{30cm}
			\vspace{.1cm}
			\begin{itemize}
				\justifying
				\item The class statement creates a new class definition.
				\item The name of the class immediately follows the keyword \textcolor{orange}{class} \\
				followed by a colon as follows
			\end{itemize}
		\end{column}
	\end{columns}
	\vspace{.5cm}
	\hspace*{5.5cm} \includegraphics[width=5cm]{figs/class.jpg}
\end{frame}

%------------------------------------------------
\begin{frame}[fragile]
	\frametitle{Classes}
	\begin{columns}[c]
		\begin{column}{30cm}
			\vspace{.05cm}
			\begin{scriptsize}
				\begin{minted}[
				bgcolor=BG,
				frame=lines,
				framesep=2mm,
				baselinestretch=1.2,
				linenos]
				{python}
				class Employee:
				    """
				    Common base class for all employees
				    """
				    empCount = 0
				
				    def __init__(self, name, salary):
				        self.name = name
				        self.salary = salary
				        Employee.empCount += 1
				
				    def displayCount(self):
				        print "Total Employee %d" % Employee.empCount
				
				    def displayEmployee(self):
				        print "Name : ", self.name,  ", Salary: ", self.salary
				\end{minted}
				\begin{minted}[
				bgcolor=BG,
				frame=lines,
				framesep=2mm,
				baselinestretch=1.2,
				linenos]
				{python}
				emp1 = Employee("Zara", 2000)
				\end{minted}
			\end{scriptsize}
		\end{column}
	\end{columns}
\end{frame}

%------------------------------------------------
\begin{frame}
	\frametitle{Garbage Collection}
	\begin{columns}[c]
		\begin{column}{30cm}
			\vspace{.1cm}
			\begin{itemize}
				\justifying
				\item Python deletes unneeded objects automatically to free the memory space.
			\end{itemize}
		\end{column}
	\end{columns}
	\vspace{.5cm}
	\hspace*{5.5cm} \includegraphics[width=5cm]{figs/garbage.jpg}
\end{frame}

%------------------------------------------------
\begin{frame}
	\frametitle{Inheritance}
	\begin{columns}[c]
		\begin{column}{30cm}
			\vspace{.1cm}
			\begin{itemize}
				\justifying
				\item Instead of starting from scratch, you can create a class by deriving it \\
				from a preexisting class by listing the parent class in parentheses \\
				after the new class name.
			\end{itemize}
		\end{column}
	\end{columns}
	\vspace{.5cm}
	\hspace*{5.5cm} \includegraphics[width=5cm]{figs/Inheritance.jpg}
\end{frame}

%------------------------------------------------
\begin{frame}[fragile]
	\frametitle{Inheritance}
	\begin{columns}[c]
		\begin{column}{30cm}
			\vspace{.05cm}
			\begin{scriptsize}
				\begin{minted}[
				bgcolor=BG,
				frame=lines,
				framesep=2mm,
				baselinestretch=1.2,
				linenos]
				{python}
				class SubClassName (ParentClass1[, ParentClass2, ...]):
				    """
				    Optional class documentation string
				    """
				    # class_suite
				\end{minted}
			\end{scriptsize}
		\end{column}
	\end{columns}
\end{frame}

%------------------------------------------------
\begin{frame}
	\frametitle{Base Overloading Methods}
	\begin{columns}[c]
		\begin{column}{30cm}
			\vspace{.1cm}
			\begin{itemize}
				\justifying
				\item \_\_init\_\_( self [,args...] ) : \textcolor{Ocean}{Constructor (with any optional arguments)}
				\item \_\_del\_\_( self ) : \textcolor{Ocean}{Destructor, deletes an object}
				\item \_\_repr\_\_( self ) : \textcolor{Ocean}{Evaluatable string representation}
				\item \_\_str\_\_( self ) : \textcolor{Ocean}{Printable string representation}
				\item \_\_lt\_\_( self, other ):
				\item \_\_le\_\_( self, other ):
				\item \_\_eq\_\_( self, other ):
				\item \_\_ne\_\_( self, other ):
				\item \_\_gt\_\_( self, other ):
				\item \_\_ge\_\_( self, other ): \\
				\textcolor{Ocean}{These are the so-called “rich comparison” methods,} \\
				\textcolor{Ocean}{and are called for comparison operators in preference to \_\_cmp\_\_() below.}
			\end{itemize}
		\end{column}
	\end{columns}
\end{frame}

%------------------------------------------------
\begin{frame}
	\frametitle{Base Overloading Methods}
	\begin{columns}[c]
		\begin{column}{30cm}
			\vspace{.1cm}
			\begin{itemize}
				\justifying
				\item \_\_cmp\_\_ ( self, x ) : \textcolor{Ocean}{Called by comparison operations if rich comparison} \\
				\textcolor{Ocean}{is not defined.} 
				\item \_\_add\_\_( self, other ):
				\item \_\_sub\_\_( self, other ):
				\item \_\_mul\_\_( self, other ):
				\item \_\_floordiv\_\_( self, other ):
				\item \_\_mod\_\_( self, other ):
				\item \_\_divmod\_\_( self, other ):
				\item \_\_pow\_\_( self, other[, modulo] ):
				\end{itemize}
		\end{column}
	\end{columns}
\end{frame}

%------------------------------------------------
\begin{frame}
	\frametitle{Base Overloading Methods}
	\begin{columns}[c]
		\begin{column}{30cm}
			\vspace{.1cm}
			\begin{itemize}
				\justifying
				\item \_\_lshift\_\_( self, other ):
				\item \_\_rshift\_\_( self, other ):
				\item \_\_and\_\_( self, other ):
				\item \_\_xor\_\_( self, other ):
				\item \_\_or\_\_( self, other ): \\
				\textcolor{Ocean}{These methods are called to implement the binary arithmetic operations} \\
				\textcolor{green}{+, -, *, //, \%, divmod(), pow(), **, \textless\textless, \textgreater\textgreater,
				\&, \textasciicircum, \textbar}
				
			\end{itemize}
		\end{column}
	\end{columns}
\end{frame}

%------------------------------------------------
\begin{frame}
	\vspace{1cm}
	\begin{Huge}
		\begin{center}
			\usebeamercolor[fg]{title}Questions?
		\end{center}
	\end{Huge}
\end{frame}

\end{document} 