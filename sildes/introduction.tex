\documentclass{beamer}

\mode<presentation> {
\usetheme{CambridgeUS}
\usecolortheme{seagull}
\usefonttheme{default}
}

\usepackage{graphicx}
\usepackage{booktabs}
\usepackage{ragged2e}
\usepackage{minted}
\usepackage{lipsum}
\usepackage[export]{adjustbox}

%----------------------------------------------------------------------------------------
%	My Customized Settings
%----------------------------------------------------------------------------------------

\definecolor{UniBlue}{RGB}{83,121,170}
\definecolor{DarkGray}{RGB}{90,90,90}
\definecolor{LightGray}{RGB}{150,150,150}
\definecolor{TextGreen}{RGB}{115,155,15}
\definecolor{Ocean}{RGB}{23,142,189}
\definecolor{BG}{RGB}{215,215,215}


\setbeamercolor{normal}{fg=DarkGray}
\setbeamercolor{title}{fg=UniBlue}
\setbeamercolor{frametitle}{fg=UniBlue}
\setbeamercolor{structure}{fg=UniBlue}
\setbeamercolor{normal text}{fg=DarkGray,bg=white}
\setbeamercolor{section number projected}{bg=UniBlue,fg=white}

\setbeamertemplate{itemize item}{\scriptsize\raise1.25pt\hbox{\donotcoloroutermaths$\blacktriangleright$}}
\setbeamertemplate{itemize subitem}{\tiny\raise1.5pt\hbox{\donotcoloroutermaths$\bullet$}}
\setbeamertemplate{itemize subsubitem}{\tiny\raise1.5pt\hbox{\donotcoloroutermaths$\blacksqaure$}}
\setbeamercolor{itemize item}{fg=darkred}
\setbeamercolor{itemize subitem}{fg=TextGreen}
\setbeamercolor{itemize subbody}{fg=LightGray}

\setbeamertemplate{enumerate subitem}{\insertenumlabel.\insertsubenumlabel}
\setbeamertemplate{enumerate subsubitem}{\insertenumlabel.\insertsubenumlabel.\insertsubsubenumlabel}
\setbeamertemplate{enumerate mini template}{\insertenumlabel}

\setbeamertemplate{navigation symbols}{}

\newcommand\VeryLargeFont{\fontsize{30}{15}\selectfont}
\newcommand\LargeFont{\fontsize{15}{15}\selectfont}
\newcommand\TinyFont{\fontsize{6}{6}\selectfont}

\setbeamertemplate{frametitle} {
  \nointerlineskip
  \begin{beamercolorbox}[sep=0.15cm,ht=1.3em,wd=\paperwidth]{frametitle}
    \vbox{}\vskip-2ex
    \strut\insertframetitle\strut
    \vskip-0.8ex
  \end{beamercolorbox}
}

\defbeamertemplate*{title page}{customized}[1][] {
  \centering
  \bigskip
  \bigskip
  \bigskip
  \usebeamercolor[fg]{title}\insertsubtitle\par
  \usebeamerfont{title}\inserttitle\par
  \usebeamerfont{subtitle}
  \bigskip
  \usebeamercolor[fg]{normal}
  \usebeamerfont{author}\insertauthor\par
  \usebeamerfont{institute}\insertinstitute\par
  \usebeamerfont{date}\insertdate\par
  \bigskip
  \bigskip
  \bigskip
  \bigskip
  \bigskip
  \usebeamercolor[fg]{titlegraphic}\inserttitlegraphic
}

%----------------------------------------------------------------------------------------
%	TITLE PAGE
%----------------------------------------------------------------------------------------
\title[Introduction]{Introduction to Python}
\author{Parham Alvani}
\institute[9231058] {
  Amirkabir University of Technology \\
  \medskip
  {\small\tt parham.alvani@gmail.com}
}
\date{May 14, 2015}
\titlegraphic{\hspace*{5cm}\includegraphics[width=2cm]{figs/aut_logo.jpeg}}

\begin{document}

\begin{frame}
\titlepage
\end{frame}

%----------------------------------------------------------------------------------------
%	PRESENTATION SLIDES
%----------------------------------------------------------------------------------------

%------------------------------------------------
\section{}
\subsection{}

%------------------------------------------------
\begin{frame}
\begin{columns}
	\begin{column}{10cm}
		\vspace{2cm}
		\begin{block}{
				\centering\textcolor{darkred}{what is python...}}
				\justifying
				Python is a high-level, interpreted, interactive and object-oriented scripting language. Python is designed to be highly readable..\\
		\end{block}
	\end{column}
\end{columns}
\vspace{.75cm}
\hspace*{8.5cm}\includegraphics[width=3cm]{figs/python.jpeg}
\end{frame}

%------------------------------------------------
\begin{frame}
	\frametitle{Environment}
	\hspace*{1.5cm}\includegraphics[width=10cm]{figs/python-console-linux.png}
\end{frame}

%------------------------------------------------
\begin{frame}
\frametitle{Variable Types}
\begin{columns}[c]
	\begin{column}{30cm}
		\vspace{.1cm}
		\begin{itemize}
			\justifying
			\item Numbers
			\item String
			\item List
			\item Tuple
			\item Dictionary
		\end{itemize}
	\end{column}
\end{columns}
	\vspace{.5cm}
	\hspace*{5.5cm} \includegraphics[width=5cm]{figs/python-types.jpg}
\end{frame}

%------------------------------------------------
\begin{frame}
	\frametitle{Numbers}
	\begin{columns}[c]
		\begin{column}{30cm}
			\vspace{.1cm}
			\begin{itemize}
				\justifying
				\item Number data types store numeric values.
				\item They are immutable data types, means that changing the value of \\
				 a number data type results in a newly allocated object.
			\end{itemize}
		\end{column}
	\end{columns}

\end{frame}

%------------------------------------------------
\begin{frame}
	\frametitle{Strings}
	\begin{columns}[c]
		\begin{column}{30cm}
			\vspace{.1cm}
			\begin{itemize}
				\justifying
				\item Strings are amongst the most popular types in Python.
				\item We can create them simply by enclosing characters in quotes.
				\item Python treats single quotes the same as double quotes.
			\end{itemize}
		\end{column}
	\end{columns}
\end{frame}

%------------------------------------------------
\begin{frame}
	\frametitle{Lists}
	\begin{columns}[c]
		\begin{column}{30cm}
			\vspace{.1cm}
			\begin{itemize}
				\justifying
				\item The most basic data structure in Python is the sequence.
				\item Each element of a sequence is assigned a number - its position or index.
				\item The first index is zero, the second index is one, and so forth.
			\end{itemize}
		\end{column}
	\end{columns}
\end{frame}

%------------------------------------------------
\begin{frame}
	\frametitle{Tuples}
	\begin{columns}[c]
		\begin{column}{30cm}
			\vspace{.1cm}
			\begin{itemize}
				\justifying
				\item A tuple is a sequence of immutable Python objects.
				\item Tuples are sequences, just like lists.
				\item The differences between tuples and lists are :
				\begin{itemize}
						\item the tuples cannot be changed unlike lists
						\item tuples use parentheses, whereas lists use square brackets.
				\end{itemize}
			\end{itemize}
		\end{column}
	\end{columns}
\end{frame}

%------------------------------------------------
\begin{frame}
	\frametitle{Dictionary}
	\begin{columns}[c]
		\begin{column}{30cm}
			\vspace{.1cm}
			\begin{itemize}
				\justifying
				\item Each key is separated from its value by a colon (:)
				\item the items are separated by commas
				\item and the whole thing is enclosed in curly braces.
			\end{itemize}
		\end{column}
	\end{columns}
\end{frame}

%------------------------------------------------
\begin{frame}
	\frametitle{Flow Control}
	\begin{columns}[c]
		\begin{column}{30cm}
			\vspace{.1cm}
			\begin{itemize}
				\justifying
				\item If-Then-Else
				\item For
				\item While
				\item Exceptions
			\end{itemize}
		\end{column}
	\end{columns}
	\vspace{.5cm}
	\hspace*{5.5cm} \includegraphics[width=5cm]{figs/flow-control.jpg}
\end{frame}

%------------------------------------------------
\begin{frame}
	\frametitle{Functions}
	\begin{columns}[c]
		\begin{column}{30cm}
			\vspace{.1cm}
			\begin{itemize}
				\justifying
				\item Function blocks begin with the keyword \textcolor{orange}{def}, \\
				followed by the function name and parentheses.
				\item Any input parameters or arguments should be placed \\
				within these parentheses.
				\item The first statement of a function can be an optional statement \\
				- the documentation string of the function or docstring.
			\end{itemize}
		\end{column}
	\end{columns}
	\vspace{.1cm}
	\hspace*{5cm} \includegraphics[width=5cm]{figs/functions.jpg}
\end{frame}

%------------------------------------------------
\begin{frame}[fragile]
	\frametitle{Functions}
	\begin{columns}[c]
		\begin{column}{30cm}
			\vspace{.1cm}
			\begin{scriptsize}
			  	\begin{minted}[
			  	bgcolor=BG,
			  	frame=lines,
			  	framesep=2mm,
			  	baselinestretch=1.2,
			  	linenos]
			  	{python}
		  			def square(x):
				  	    return x * x
		  		\end{minted}
		  		\begin{minted}[
		  		bgcolor=BG,
		  		frame=lines,
		  		framesep=2mm,
		  		baselinestretch=1.2,
		  		linenos]
		  		{python}
			  		def hello():
		  			    return "Hello"
			  	\end{minted}
			  	\begin{minted}[
			  	bgcolor=BG,
			  	frame=lines,
			  	framesep=2mm,
			  	baselinestretch=1.2,
			  	linenos]
			  	{python}
			  	def printme( str ):
			  	    "This prints a passed string into this function"
			  	    print str
			  	    return
			  	\end{minted}
			\end{scriptsize}
		\end{column}
	\end{columns}
\end{frame}

%------------------------------------------------
\begin{frame}
	\frametitle{Classes}
	\begin{columns}[c]
		\begin{column}{30cm}
			\vspace{.1cm}
			\begin{itemize}
				\justifying
				\item The class statement creates a new class definition.
				\item The name of the class immediately follows the keyword \textcolor{orange}{class} \\
				followed by a colon as follows
			\end{itemize}
		\end{column}
	\end{columns}
	\vspace{.5cm}
	\hspace*{5.5cm} \includegraphics[width=5cm]{figs/class.jpg}
\end{frame}

%------------------------------------------------
\begin{frame}[fragile]
	\frametitle{Classes}
	\begin{columns}[c]
		\begin{column}{30cm}
			\vspace{.05cm}
			\begin{scriptsize}
				\begin{minted}[
				bgcolor=BG,
				frame=lines,
				framesep=2mm,
				baselinestretch=1.2,
				linenos]
				{python}
				class Employee:
				    """
				    Common base class for all employees
				    """
				    empCount = 0
				
				    def __init__(self, name, salary):
				        self.name = name
				        self.salary = salary
				        Employee.empCount += 1
				
				    def displayCount(self):
				        print "Total Employee %d" % Employee.empCount
				
				    def displayEmployee(self):
				        print "Name : ", self.name,  ", Salary: ", self.salary
				\end{minted}
				\begin{minted}[
				bgcolor=BG,
				frame=lines,
				framesep=2mm,
				baselinestretch=1.2,
				linenos]
				{python}
				emp1 = Employee("Zara", 2000)
				\end{minted}
			\end{scriptsize}
		\end{column}
	\end{columns}
\end{frame}

%------------------------------------------------
\begin{frame}
	\frametitle{Garbage Collection}
	\begin{columns}[c]
		\begin{column}{30cm}
			\vspace{.1cm}
			\begin{itemize}
				\justifying
				\item Python deletes unneeded objects automatically to free the memory space.
			\end{itemize}
		\end{column}
	\end{columns}
	\vspace{.5cm}
	\hspace*{5.5cm} \includegraphics[width=5cm]{figs/garbage.jpg}
\end{frame}

%------------------------------------------------
\begin{frame}
	\frametitle{Inheritance}
	\begin{columns}[c]
		\begin{column}{30cm}
			\vspace{.1cm}
			\begin{itemize}
				\justifying
				\item Instead of starting from scratch, you can create a class by deriving it \\
				from a preexisting class by listing the parent class in parentheses \\
				after the new class name.
			\end{itemize}
		\end{column}
	\end{columns}
	\vspace{.5cm}
	\hspace*{5.5cm} \includegraphics[width=5cm]{figs/Inheritance.jpg}
\end{frame}

%------------------------------------------------
\begin{frame}[fragile]
	\frametitle{Inheritance}
	\begin{columns}[c]
		\begin{column}{30cm}
			\vspace{.05cm}
			\begin{scriptsize}
				\begin{minted}[
				bgcolor=BG,
				frame=lines,
				framesep=2mm,
				baselinestretch=1.2,
				linenos]
				{python}
				class SubClassName (ParentClass1[, ParentClass2, ...]):
				    """
				    Optional class documentation string
				    """
				    # class_suite
				\end{minted}
			\end{scriptsize}
		\end{column}
	\end{columns}
\end{frame}

%------------------------------------------------
\begin{frame}
	\frametitle{Base Overloading Methods}
	\begin{columns}[c]
		\begin{column}{30cm}
			\vspace{.1cm}
			\begin{itemize}
				\justifying
				\item \_\_init\_\_( self [,args...] ) : \textcolor{Ocean}{Constructor (with any optional arguments)}
				\item \_\_del\_\_( self ) : \textcolor{Ocean}{Destructor, deletes an object}
				\item \_\_repr\_\_( self ) : \textcolor{Ocean}{Evaluatable string representation}
				\item \_\_str\_\_( self ) : \textcolor{Ocean}{Printable string representation}
				\item \_\_lt\_\_( self, other ):
				\item \_\_le\_\_( self, other ):
				\item \_\_eq\_\_( self, other ):
				\item \_\_ne\_\_( self, other ):
				\item \_\_gt\_\_( self, other ):
				\item \_\_ge\_\_( self, other ): \\
				\textcolor{Ocean}{These are the so-called “rich comparison” methods,} \\
				\textcolor{Ocean}{and are called for comparison operators in preference to \_\_cmp\_\_() below.}
			\end{itemize}
		\end{column}
	\end{columns}
\end{frame}

%------------------------------------------------
\begin{frame}
	\frametitle{Base Overloading Methods}
	\begin{columns}[c]
		\begin{column}{30cm}
			\vspace{.1cm}
			\begin{itemize}
				\justifying
				\item \_\_cmp\_\_ ( self, x ) : \textcolor{Ocean}{Called by comparison operations if rich comparison} \\
				\textcolor{Ocean}{is not defined.} 
				\item \_\_add\_\_( self, other ):
				\item \_\_sub\_\_( self, other ):
				\item \_\_mul\_\_( self, other ):
				\item \_\_floordiv\_\_( self, other ):
				\item \_\_mod\_\_( self, other ):
				\item \_\_divmod\_\_( self, other ):
				\item \_\_pow\_\_( self, other[, modulo] ):
				\end{itemize}
		\end{column}
	\end{columns}
\end{frame}

%------------------------------------------------
\begin{frame}
	\frametitle{Base Overloading Methods}
	\begin{columns}[c]
		\begin{column}{30cm}
			\vspace{.1cm}
			\begin{itemize}
				\justifying
				\item \_\_lshift\_\_( self, other ):
				\item \_\_rshift\_\_( self, other ):
				\item \_\_and\_\_( self, other ):
				\item \_\_xor\_\_( self, other ):
				\item \_\_or\_\_( self, other ): \\
				\textcolor{Ocean}{These methods are called to implement the binary arithmetic operations} \\
				\textcolor{green}{+, -, *, //, \%, divmod(), pow(), **, \textless\textless, \textgreater\textgreater,
				\&, \textasciicircum, \textbar}
				
			\end{itemize}
		\end{column}
	\end{columns}
\end{frame}

%------------------------------------------------
\begin{frame}
\vspace{1cm}
\Huge{\centerline{\usebeamercolor[fg]{title}A Brief History of}}
\Huge{\centerline{\usebeamercolor[fg]{title}Data Management!}}
\end{frame}

%------------------------------------------------
\begin{frame}
\frametitle{4000 B.C}
\begin{columns}[c] 
\column{30em}
\begin{itemize}
  \justifying
  \item Manual recording
  \item From tablets to papyrus, to parchment, and then to paper
\end{itemize}
\end{columns}
\vspace{0.5cm}
\hspace*{3.5cm}\includegraphics[width=5cm]{figs/egyptian.pdf}
\end{frame}

%------------------------------------------------
\begin{frame}
\frametitle{1450}
\begin{columns}[c] 
\column{30em}
\begin{itemize}
  \justifying
  \item Gutenberg's printing press
\end{itemize}
\end{columns}
\vspace{0.75cm}
\hspace*{4.5cm}\includegraphics[width=4cm]{figs/gutenberg.pdf}
\end{frame}

%------------------------------------------------
\begin{frame}
\frametitle{1800's - 1940's}
\begin{columns}[c] 
\column{30em}
\begin{itemize}
  \justifying
  \item Punched cards (no fault-tolerance)
  \item Binary data
  \item 1890: US census
  \item 1911: IBM appeared
\end{itemize}
\end{columns}
\vspace{0.5cm}
\begin{columns}[c] 
\column{.4\textwidth}
\includegraphics[width=6cm]{figs/punch_card.pdf}
\column{.4\textwidth}
\hspace*{1cm}\includegraphics[width=4cm]{figs/ibm.pdf}
\end{columns}
\end{frame}

%------------------------------------------------
\begin{frame}
\frametitle{1940's - 1970's}
\begin{columns}[c] 
\column{30em}
\begin{itemize}
  \justifying
  \item Magnetic tapes
  \item Batch transaction processing
  \item File-oriented record processing model (e.g., COBOL)
  \item Hierarchical DBMS (one-to-many)
  \item Network DBMS (many-to-many)
\end{itemize}
\end{columns}
\vspace{0.75cm}
\begin{columns}[c] 
\column{.2\textwidth}
\includegraphics[width=4cm]{figs/tape.pdf}
\column{.4\textwidth}
\hspace*{1cm}\includegraphics[width=4cm]{figs/bombers.pdf}
\end{columns}
\end{frame}

%------------------------------------------------
\begin{frame}
\frametitle{1980's}
\begin{columns}[c] 
\column{30em}
\begin{itemize}
  \justifying
  \item Relational DBMS (tables) and SQL
  \item ACID
  \item Client-server computing
  \item Parallel processing
\end{itemize}
\end{columns}
\vspace{0.75cm}
\hspace*{2.8cm}\includegraphics[width=6cm]{figs/acid.pdf}
\end{frame}

%------------------------------------------------
\begin{frame}
\frametitle{1990's - 2000's}
\begin{columns}[c] 
\column{30em}
\begin{itemize}
  \justifying
  \item The Internet...
\end{itemize}
\end{columns}
\vspace{0.75cm}
\hspace*{2.5cm}\includegraphics[width=7cm]{figs/internet.pdf}
\end{frame}

%------------------------------------------------
\begin{frame}
\frametitle{2010's}
\begin{columns}[c] 
\column{30em}
\begin{itemize}
  \justifying
  \item NoSQL: BASE instead of ACID
  \item Big Data
\end{itemize}
\end{columns}
\vspace{0.75cm}
\hspace*{3cm}\includegraphics[width=6cm]{figs/nosql.pdf}
\end{frame}

%------------------------------------------------
%\begin{frame}
%\frametitle{too BIG to IGNORE}
%\vspace{2cm}
%\begin{columns}
%\column{30em}
%\begin{block}{}
%\centering
%Everyone talks about it, nobody really knows how to do it, everyone thinks everyone else is doing it, so everyone claims they are doing it.
%\vspace{.1cm}
%\hspace*{9cm}\footnotesize{- Dan Ariely}
%\end{block}
%\end{columns}
%\vspace{.5cm}
%\hspace*{10cm}\includegraphics[width=1.5cm]{figs/toobig.pdf}\\
%\end{frame}

%------------------------------------------------
\begin{frame}
\frametitle{Big Data}
\begin{columns}
\column{30em}
\begin{itemize}\itemsep2em
  \justifying
  \item In recent years we have witnessed a \textcolor{Ocean}{dramatic increase} in available data.
  \item For example, the \textcolor{Ocean}{number of web pages} indexed by Google, which were around \textcolor{TextGreen}{one million} in 1998, have exceeded \textcolor{TextGreen}{one trillion} in 2008, and its expansion is accelerated by appearance of the social networks.
\end{itemize}
\end{columns}
\vspace{0.35cm}
\hspace*{8.7cm}\includegraphics[width=3cm]{figs/zeroone.pdf}
\end{frame}

%------------------------------------------------
\begin{frame}
\frametitle{Big Data Definition}
\begin{columns}[c] 
\column{.65\textwidth}
\begin{itemize}\itemsep2em
  \justifying
  \item \textcolor{darkred}{Big Data} refers to datasets and flows \textcolor{Ocean}{large enough} that has outpaced our capability to \textcolor{TextGreen}{store}, \textcolor{TextGreen}{process}, \textcolor{TextGreen}{analyze}, and \textcolor{TextGreen}{understand}.
\end{itemize}

\column{.25\textwidth}
\includegraphics[width=3cm]{figs/big-data.pdf}
\end{columns}
\end{frame}

%------------------------------------------------
\begin{frame}
\frametitle{The Four Dimensions of Big Data}
\begin{columns}[c] 
\column{.6\textwidth}
\begin{itemize}\itemsep1em
  \justifying
  \item \textcolor{Ocean}V\textcolor{darkred}{olume}: data \textcolor{TextGreen}{size}
  \item \textcolor{Ocean}V\textcolor{darkred}{elocity}: data \textcolor{TextGreen}{generation rate}
  \item \textcolor{Ocean}V\textcolor{darkred}{ariety}: data \textcolor{TextGreen}{heterogeneity}
  \item \textcolor{Ocean}V\textcolor{darkred}{eracity}: \textcolor{TextGreen}{uncertainty} of accuracy and \\ authenticity of data
\end{itemize}

\column{.32\textwidth}
\includegraphics[width=3.5cm]{figs/vs.pdf}
\end{columns}
\end{frame}

%------------------------------------------------
\begin{frame}
\frametitle{Big Data Market Driving Factors}
\begin{columns}[c] 
\column{.4\textwidth}
\begin{itemize}\itemsep2em
  \justifying
  \item Mobile devices
  \item Internet of Things (IoT)
  \item Cloud computing
  \item Open source initiatives
\end{itemize}
\column{.4\textwidth}
\includegraphics[width=5cm]{figs/driving_factors.pdf}
\end{columns}
\end{frame}

%------------------------------------------------
%\begin{frame}
%\frametitle{Big Data Applications}
%\begin{columns}[c] 
%\column{.55\textwidth}
%\begin{itemize}\itemsep1em
%  \justifying
%  \item Telecommunication and the Internet
%  \item Finance
%  \item Healthcare
%  \item Education
%  \item Transport
%\end{itemize}
%\column{.35\textwidth}
%\includegraphics[width=4cm]{figs/applications.pdf}
%\end{columns}
%\end{frame}


%------------------------------------------------
\begin{frame}
\vspace{1cm}
\Huge{\centerline{\usebeamercolor[fg]{title}The Big Data Stack!}}
\end{frame}

%------------------------------------------------
\begin{frame}
\frametitle{Big Data Analytics Stack}
\hspace*{3cm}\includegraphics[width=6cm]{figs/stack.pdf}
\end{frame}

%------------------------------------------------
\begin{frame}
\frametitle{Big Data - Storage (Filesystem)}
\begin{columns}
\column{30em}
\begin{itemize}\itemsep1em
  \item Traditional filesystems are not well-designed for large-scale data processing systems. 
  \item \textcolor{TextGreen}{Efficiency} has a higher priority than other features, e.g., directory service. 
  %\item Data tends to be written and read in \textcolor{Ocean}{large batches} at once. 
  \item Massive size of data tends to store it across \textcolor{Ocean}{multiple machines} in a distributed way. 
  \item HDFS, Amazon S3, ...
\end{itemize}
\end{columns}
\vspace{1.2cm}
\hspace*{10cm}\includegraphics[width=2cm]{figs/stack_storage.pdf}
\end{frame}

%------------------------------------------------
\begin{frame}
\frametitle{Big Data - Database}
\begin{columns}
\column{30em}
\begin{itemize}\itemsep1em
  \justifying
  \item Relational Databases Management Systems \textcolor{Ocean}{(RDMS)} were \textcolor{TextGreen}{not} designed to be distributed.
  \item \textcolor{darkred}{NoSQL} databases \textcolor{Ocean}{relax} one or more of the \textcolor{Ocean}{ACID} properties: \textcolor{TextGreen}{BASE}
  \item Different data models: \textcolor{Ocean}{key/value}, \textcolor{Ocean}{column-family}, \textcolor{Ocean}{graph}, \textcolor{Ocean}{document}.
  \item Dynamo, Scalaris, BigTable, Hbase, Cassandra, MongoDB, Voldemort, Riak, Neo4J, ...
\end{itemize}
\end{columns}
\vspace{1.83cm}
\hspace*{10cm}\includegraphics[width=2cm]{figs/stack_database.pdf}
\end{frame}

%------------------------------------------------
\begin{frame}
\frametitle{Big Data - Resource Management}
\begin{columns}
\column{30em}
\begin{itemize}\itemsep1em
  \justifying
  \item Different frameworks require different \textcolor{Ocean}{computing resources}.
  \item Large organizations need the ability to \textcolor{Ocean}{share data and resources} between multiple frameworks. 
  \item \textcolor{darkred}{Resource management} share resources in a cluster between \textcolor{Ocean}{multiple frameworks} while providing resource \textcolor{TextGreen}{isolation}.
  \item Mesos, YARN, Quincy, ...
\end{itemize}
\end{columns}
\vspace{1.8cm}
\hspace*{10cm}\includegraphics[width=2cm]{figs/stack_job.pdf}
\end{frame}

%------------------------------------------------
\begin{frame}
\frametitle{Big Data - Execution Engine}
\begin{columns}
\column{30em}
\begin{itemize}\itemsep1em
  \justifying
  \item \textcolor{Ocean}{Scalable} and \textcolor{Ocean}{fault tolerance} parallel data processing on clusters of unreliable machines.
  %\item \textcolor{darkred}{Fine-grained} scheduling model is desirable, such that computations are operated on the \textcolor{Ocean}{same node} with the data needed: \textcolor{TextGreen}{lowering the cost}.
  \item Data-parallel \textcolor{Ocean}{programming model} for clusters of commodity machines.
  \item MapReduce, Spark, Stratosphere, Dryad, Hyracks, ...
\end{itemize}
\end{columns}
\vspace{2.55cm}
\hspace*{10cm}\includegraphics[width=2cm]{figs/stack_execution.pdf}
\end{frame}

%------------------------------------------------
\begin{frame}
\frametitle{Big Data - Query/Scripting Language}
\begin{columns}
\column{30em}
\begin{itemize}\itemsep1em
  \justifying
  \item \textcolor{Ocean}{Low-level} programming of execution engines, e.g., MapReduce, is \textcolor{TextGreen}{not} easy for end users.
  \item Need \textcolor{Ocean}{high-level} language to improve the query capabilities of execution engines.
  \item It translates \textcolor{TextGreen}{user-defined} functions to \textcolor{TextGreen}{low-level} API of the execution engines.
  \item Pig, Hive, Shark, Meteor, DryadLINQ, SCOPE, ...
\end{itemize}
\end{columns}
\vspace{1.33cm}
\hspace*{10cm}\includegraphics[width=2cm]{figs/stack_query.pdf}
\end{frame}

%------------------------------------------------
\begin{frame}
\frametitle{Big Data - Stream Processing}
\begin{columns}
\column{30em}
\begin{itemize}\itemsep1em
  \justifying
  \item Providing users with \textcolor{Ocean}{fresh} and \textcolor{Ocean}{low latency} results.
  \item Database Management Systems (\textcolor{TextGreen}{DBMS}) vs. Stream Processing Systems (\textcolor{TextGreen}{SPS})
  \vspace{0.3cm}
  \begin{columns}[c] 
  \column{.3\textwidth}
  \includegraphics[width=3cm,valign=t]{figs/dbms.pdf}
  \column{.3\textwidth}
  \includegraphics[width=3cm,valign=t]{figs/sps.pdf}
  \end{columns}
  \item Storm, S4, SEEP, D-Stream, Naiad, ...
\end{itemize}
\end{columns}
\vspace{0.62cm}
\hspace*{10cm}\includegraphics[width=2cm]{figs/stack_streaming.pdf}
\end{frame}

%------------------------------------------------
\begin{frame}
\frametitle{Big Data - Graph Processing}
\begin{columns}
\column{30em}
\begin{itemize}\itemsep1em
  \justifying
  \item Many problems are expressed using \textcolor{TextGreen}{graphs}: sparse \textcolor{Ocean}{computational dependencies}, and \textcolor{Ocean}{multiple iterations} to converge.
  \item Data-parallel frameworks, such as MapReduce, are not ideal for these problems: \textcolor{Ocean}{slow}
  \item Graph processing frameworks are \textcolor{Ocean}{optimized} for graph-based problems.
  \item Pregel, Giraph, GraphX, GraphLab, PowerGraph, GraphChi, ...
\end{itemize}
\end{columns}
\vspace{1.33cm}
\hspace*{10cm}\includegraphics[width=2cm]{figs/stack_graph.pdf}
\end{frame}

%------------------------------------------------
\begin{frame}
\frametitle{Big Data - Machine Learning}
\begin{columns}
\column{30em}
\begin{itemize}\itemsep1em
  \justifying
  \item Implementing and consuming machine learning techniques at scale are \textcolor{Ocean}{difficult tasks} for developers and end users.
  \item There exist platforms that address it by providing scalable machine-learning and data mining libraries.
  \item Mahout, MLBase, SystemML, Ricardo, Presto, ...
\end{itemize}
\end{columns}
\vspace{2.67cm}
\hspace*{10cm}\includegraphics[width=2cm]{figs/stack_ml.pdf}
\end{frame}

%------------------------------------------------
\begin{frame}
\frametitle{Hadoop Big Data Analytics Stack}
\hspace*{4.7cm}\includegraphics[width=2.5cm]{figs/hadoop.pdf}
\vspace{0.5cm}
\hspace*{3cm}\includegraphics[width=6cm]{figs/stack_hadoop.pdf}
\end{frame}

%------------------------------------------------
\begin{frame}
\frametitle{Stratosphere Big Data Analytics Stack}
\hspace*{4.5cm}\includegraphics[width=2.5cm]{figs/stratosphere.pdf}
\vspace{0.5cm}
\hspace*{3cm}\includegraphics[width=6cm]{figs/stack_stratosphere.pdf}
\end{frame}

%------------------------------------------------
\begin{frame}
\frametitle{Spark Big Data Analytics Stack}
\hspace*{5cm}\includegraphics[width=2cm]{figs/spark.pdf}
\vspace{0.5cm}
\hspace*{3cm}\includegraphics[width=6cm]{figs/stack_spark.pdf}
\end{frame}

%------------------------------------------------
\begin{frame}
\frametitle{Summary}
\hspace*{3cm}\includegraphics[width=6cm]{figs/summary.pdf}
\end{frame}

%------------------------------------------------
\begin{frame}
\vspace{1cm}
\Huge{\centerline{\usebeamercolor[fg]{title}Questions?}}
\end{frame}

\end{document} 