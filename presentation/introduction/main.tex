\documentclass{../py-lecture}

\subtitle{Introduction}

\begin{document}

\begin{frame}
  \titlepage{}
\end{frame}
\begin{frame}
  \frametitle{Outline}
  \tableofcontents{}
\end{frame}

\section{Introduction}

\begin{frame}
  \begin{block}{
      \centering\textcolor{darkgray}{What is python\ldots}}
      Python is a high-level, interpreted, interactive and object-oriented scripting language. Python is designed to be highly readable.
  \end{block}
  \vspace{.75cm}
  \hspace*{8.5cm}\includegraphics[width=3cm]{img/python.jpeg}
\end{frame}

\begin{frame}
	\frametitle{Environment}
	\includegraphics[width=\linewidth]{img/python-console-linux.png}
\end{frame}

\begin{frame}
	\frametitle{Variable Types}
  \begin{columns}
    \begin{column}{0.5\textwidth}
      \begin{itemize}
        \item Numbers
        \item String
        \item List
        \item Tuple
        \item Dictionary
      \end{itemize}
    \end{column}
    \begin{column}{0.5\textwidth}
	    \includegraphics[width=\linewidth]{img/python-types.jpg}
    \end{column}
  \end{columns}
\end{frame}

\begin{frame}
	\frametitle{Numbers}
  \begin{itemize}
    \item Number data types store numeric values.
    \item They are immutable data types, means that changing the value of
      a number data type results in a newly allocated object.
  \end{itemize}
\end{frame}

\begin{frame}
	\frametitle{Strings}
  \begin{itemize}
    \item Strings are among the most popular types in Python.
    \item We can create them simply by enclosing characters in quotes.
    \item Python treats single quotes the same as double quotes.
  \end{itemize}
\end{frame}

\begin{frame}
	\frametitle{Lists}
  \begin{itemize}
    \item The most basic data structure in Python is the sequence.
    \item Each element of a sequence is assigned a number - its position or index.
    \item The first index is zero, the second index is one, and so forth.
  \end{itemize}
	\centering\includegraphics[width=.4\textwidth]{img/list.jpg}	
\end{frame}

\begin{frame}
	\frametitle{Tuples}
  \begin{itemize}
    \item A tuple is a sequence of immutable Python objects.
    \item Tuples are sequences, just like lists.
    \item The differences between tuples and lists are:
    \begin{itemize}
        \item the tuples cannot be changed unlike lists
        \item tuples use parentheses, whereas lists use square brackets.
    \end{itemize}
  \end{itemize}
	\centering\includegraphics[width=.4\textwidth]{img/tuple.jpg}
\end{frame}

\begin{frame}
	\frametitle{Dictionary}
  \begin{itemize}
    \item Each key is separated from its value by a colon (:)
    \item The items are separated by commas
    \item The whole thing is enclosed in curly braces.
  \end{itemize}
	\centering\includegraphics[width=.4\textwidth]{img/dictionary.jpg}
\end{frame}

\section{Flow Control}

\begin{frame}
	\frametitle{Flow Control}
  \begin{itemize}
    \item \mint{python}|if else|
    \item \mint{python}|for|
    \item \mint{python}|while|
    \item \mint{python}|exceptions|
  \end{itemize}
	\centering\includegraphics[width=.4\textwidth]{img/flow-control.jpg}
\end{frame}

\section{Functions}

\begin{frame}
	\frametitle{Functions}
  \begin{itemize}
    \item Function blocks begin with the keyword \textcolor{Orange}{def},
    followed by the function name and parentheses.
    \item Any input parameters or arguments should be placed
    within these parentheses.
    \item The first statement of a function can be an optional statement -
    the documentation string of the function or docstring.
  \end{itemize}
	\centering \includegraphics[width=.4\textwidth]{img/functions.jpg}
\end{frame}

\begin{frame}[fragile]
	\frametitle{Functions (Cont'd)}
  \begin{minted}[bgcolor=Black]{python}
def square(x):
    return x * x
  \end{minted}
  \begin{minted}[bgcolor=Black]{python}
def hello():
    return "Hello"
  \end{minted}
  \begin{minted}[bgcolor=Black]{python}
def printme( statement ):
    "This prints a passed string into this function"
    print statement
    return
  \end{minted}
\end{frame}

\begin{frame}[fragile]
	\frametitle{Lambda}
  \begin{itemize}
    \item Python supports simple anonymous functions through the lambda form.
    \item The executable body of the lambda must be an expression
    and can't be a statement, which is a restriction that limits its utility.
  \end{itemize}
  \begin{minted}[bgcolor=Black]{python}
foo = lambda x: x * x
  \end{minted}
	\centering \includegraphics[width=.3\textwidth]{img/lambda.png}
\end{frame}

\section{Classes}

\begin{frame}
	\frametitle{Classes}
  \begin{itemize}
    \item The class statement creates a new class definition.
    \item The name of the class immediately follows the keyword \textcolor{Orange}{class}
    followed by a colon as follows
  \end{itemize}
	\centering \includegraphics[width=.3\textwidth]{img/class.jpg}
\end{frame}

\begin{frame}[fragile]
	\frametitle{Classes (Cont'd)}
  \inputminted[bgcolor=Black,fontsize=\scriptsize,lastline=18]{python}{./src/employee.py}
\end{frame}

\begin{frame}[fragile]
	\frametitle{Classes (Cont'd)}
  \inputminted[bgcolor=Black,fontsize=\scriptsize,firstline=19]{python}{./src/employee.py}
\end{frame}

\end{document}
