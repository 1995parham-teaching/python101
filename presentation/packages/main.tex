\documentclass{../py-lecture}

\subtitle{Packages}

\begin{document}

\begin{frame}
  \titlepage{}
\end{frame}
\begin{frame}
  \frametitle{Outline}
  \tableofcontents{}
\end{frame}

\section{Input/Output}

\begin{frame}[fragile]
	\frametitle{Keyboard Input}
    \begin{itemize}
      \item The input() function reads a line from sys.stdin and
        returns it with the trailing newline stripped.
    \end{itemize}
    \begin{minted}[bgcolor=Black]{python}
name = input("Enter your input: ")
print("Received input is : ", name)
    \end{minted}
\end{frame}

\begin{frame}
	\frametitle{File IO}
  \begin{itemize}
    \item The open Function
    \item The file Object
    \item The \mintinline{python}|close()| Method
    \item The \mintinline{python}|write()| Method
    \item The \mintinline{python}|read()| Method
    \item Check
    \href{https://docs.python.org/3/library/functions.html}{here}
    for more details and functions.
  \end{itemize}
\end{frame}

\begin{frame}[fragile]
	\frametitle{The open Function}
  \begin{itemize}
    \item Before you can read or write a file,
    you have to open it using Python's built-in open() function.
    \item This function creates a file object, which would be utilized
    to call other support methods associated with it.
  \end{itemize}
  \begin{minted}[bgcolor=Black]{python}
file_object = open(file_name [, access_mode])
	\end{minted}
  \begin{minted}[bgcolor=Black]{python}
fo = open("foo.txt", "w", encoding='utf-8')
	\end{minted}
\end{frame}

\begin{frame}[fragile]
	\frametitle{The write() Method}
  \begin{itemize}
    \item The write() method writes any string to an open file.
  \end{itemize}
  \begin{minted}[bgcolor=Black]{python}
file_object.write(string)
  \end{minted}
  \begin{minted}[bgcolor=Black]{python}
# Open a file
fo = open("foo.txt", "wb")
fo.write( "Python is a great language.\nYeah its great!!\n")

# Close opend file
fo.close()
  \end{minted}
\end{frame}

\begin{frame}[fragile]
	\frametitle{The read() Method}
  \begin{itemize}
    \item The read() method reads a string from an open file.
  \end{itemize}
  \begin{minted}[bgcolor=Black]{python}
file_object.read([count])
  \end{minted}
  \begin{minted}[bgcolor=Black]{python}
# Open a file
fo = open("foo.txt", "r+", encoding="utf-8")
str = fo.read(10);
print("Read String is : ", str)
# Close opend file
fo.close()
	\end{minted}
\end{frame}

\begin{frame}[fragile]
	\frametitle{The readline() Method}
  \begin{itemize}
    \item Read one entire line from the file.
    \item A trailing newline character is kept in the string.
    \item If the size argument is present and non-negative,
    it is a maximum byte count (including the trailing newline)
    and an incomplete line may be returned.
  \end{itemize}
  \begin{minted}[bgcolor=Black]{python}
file_object.readline([size])
  \end{minted}
\end{frame}

\begin{frame}[fragile]
	\frametitle{The readlines() Method}
  \begin{itemize}
    \item Read until EOF using readline() and return a \\
      list containing the lines thus read.
  \end{itemize}
  \begin{minted}[bgcolor=Black]{python}
file_object.readlines([sizehint])
  \end{minted}
\end{frame}

\end{document}
