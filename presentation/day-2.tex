%----------------------------------------------------------------------------------------
%	PRESENTATION SLIDES
%----------------------------------------------------------------------------------------

%------------------------------------------------
\section{}
\subsection{}

%------------------------------------------------
\begin{frame}
	\frametitle{Modules}
	\begin{columns}[c]
		\begin{column}{30cm}
			\vspace{.1cm}
			\begin{itemize}
				\justifying
				\item A module allows you to logically organize your Python code.
				\item Grouping related code into a module makes the code easier to understand and use
			\end{itemize}
		\end{column}
	\end{columns}
\end{frame}

%------------------------------------------------
\begin{frame}[fragile]
	\frametitle{Modules}
	\begin{columns}[c]
		\begin{column}{30cm}
			\vspace{.1cm}
			\begin{scriptsize}
				\begin{minted}[
				bgcolor=BG,
				frame=lines,
				framesep=2mm,
				baselinestretch=1.2,
				linenos]
				{python}
				def print_func( par ):
				    print "Hello : ", par
				    return
				\end{minted}
				\begin{minted}[
				bgcolor=BG,
				frame=lines,
				framesep=2mm,
				baselinestretch=1.2,
				linenos]
				{python}
				# Import module support
				import support
							
				# Now you can call defined function that module as follows
				support.print_func("Zara")
				\end{minted}
			\end{scriptsize}
		\end{column}
	\end{columns}
\end{frame}

%------------------------------------------------
\begin{frame}
	\frametitle{Modules}
	\begin{columns}[c]
		\begin{column}{30cm}
			\vspace{.1cm}
			\begin{itemize}
				\justifying
				\item Python's from statement lets you import specific attributes from a \\
				 module into the current namespace.
				\item Locating Modules
				\begin{itemize}
					\item The current directory.
					\item Python then searches each directory in the shell variable PYTHONPATH.
					\item If all else fails, Python checks the default path. \\
					On UNIX, this default path is normally /usr/local/lib/python/.
				\end{itemize}
			\end{itemize}
		\end{column}
	\end{columns}
\end{frame}

%------------------------------------------------
\begin{frame}[fragile]
	\frametitle{Socket Programming}
	\begin{columns}[c]
		\begin{column}{30cm}
			\vspace{.1cm}
			\begin{itemize}
				\justifying
				\item Python provides two levels of access to network services
				\item To create a socket, you must use the socket.socket()
				\item See
				\textcolor{blue}{\href{https://docs.python.org/3.0/library/socket.html}{here}}
				for more details and functions.
				\item \textcolor{green}{socket\_family}: This is either AF\_UNIX or AF\_INET.
				\item \textcolor{green}{socket\_type}: This is either SOCK\_STREAM or SOCK\_DGRAM.
				\item \textcolor{green}{protocol}: This is usually left out, defaulting to 0.
			\end{itemize}
			\begin{scriptsize}
				\begin{minted}[
					bgcolor=BG,
					frame=lines,
					framesep=2mm,
					baselinestretch=1.2,
					linenos]
					{python}
					s = socket.socket (socket_family, socket_type, protocol=0)
				\end{minted}
			\end{scriptsize}
		\end{column}
	\end{columns}
\end{frame}

%------------------------------------------------
\begin{frame}[fragile]
	\frametitle{Socket Programming}
	\begin{columns}[c]
		\begin{column}{30cm}
			\vspace{.1cm}
			\begin{scriptsize}
				\begin{minted}[
				bgcolor=BG,
				frame=lines,
				framesep=2mm,
				baselinestretch=1.2,
				linenos]
				{python}
				import socket               # Import socket module
				
				s = socket.socket()         # Create a socket object
				host = socket.gethostname() # Get local machine name
				port = 12345                # Reserve a port for your service.
				s.bind((host, port))        # Bind to the port
				
				s.listen(5)                 # Now wait for client connection.
				while True:
				    c, addr = s.accept()     # Establish connection with client.
				    print('Got connection from', addr)
				    c.send('Thank you for connecting')
				    c.close()                # Close the connection
				\end{minted}
			\end{scriptsize}
		\end{column}
	\end{columns}
\end{frame}

%------------------------------------------------
\begin{frame}[fragile]
	\frametitle{Socket Programming}
	\begin{columns}[c]
		\begin{column}{30cm}
			\vspace{.1cm}
			\begin{scriptsize}
				\begin{minted}[
				bgcolor=BG,
				frame=lines,
				framesep=2mm,
				baselinestretch=1.2,
				linenos]
				{python}
				import socket               # Import socket module
				
				s = socket.socket()         # Create a socket object
				host = socket.gethostname() # Get local machine name
				port = 12345                # Reserve a port for your service.
				
				s.connect((host, port))
				print s.recv(1024)
				s.close                     # Close the socket when done
				\end{minted}
			\end{scriptsize}
		\end{column}
	\end{columns}
\end{frame}

%------------------------------------------------
\begin{frame}[fragile]
	\frametitle{Multithreading}
	\begin{columns}[c]
		\begin{column}{30cm}
			\begin{scriptsize}
				\begin{minted}[
				bgcolor=BG,
				frame=lines,
				framesep=1mm,
				baselinestretch=1.2,
				linenos]
				{python}
				class myThread (threading.Thread):
				    def __init__(self, threadID, name, counter):
				        threading.Thread.__init__(self)
				        self.threadID = threadID
				        self.name = name
				        self.counter = counter
				    def run(self):
				        print("Starting " + self.name)
				        print_time(self.name, self.counter, 5)
				        print("Exiting " + self.name)
				
				def print_time(threadName, delay, counter):
				    while counter:
				        time.sleep(delay)
				        print("%s: %s" % (threadName, time.ctime(time.time())))
				        counter -= 1
				\end{minted}
			\end{scriptsize}	
		\end{column}
	\end{columns}
\end{frame}

%------------------------------------------------
\begin{frame}[fragile]
	\frametitle{Multithreading}
	\begin{columns}[c]
		\begin{column}{30cm}
			\begin{scriptsize}
				\begin{minted}[
				bgcolor=BG,
				frame=lines,
				framesep=1mm,
				baselinestretch=1.2,
				linenos]
				{python}
				# Create new threads
				thread1 = myThread(1, "Thread-1", 1)
				thread2 = myThread(2, "Thread-2", 2)
				
				# Start new Threads
				thread1.start()
				thread2.start()
				
				print("Exiting Main Thread")
				\end{minted}
			\end{scriptsize}	
		\end{column}
	\end{columns}
\end{frame}

%------------------------------------------------
\begin{frame}
	\vspace{1cm}
	\begin{Huge}
		\begin{center}
			\usebeamercolor[fg]{title}Questions?
		\end{center}
	\end{Huge}
\end{frame}

\end{document} 
